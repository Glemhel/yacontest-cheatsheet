\begin{problem}{Запрос к сервису (get divmod)}{stdin}{stdout}{1 second(s)}{64 MiB}
Напишите программу с использованием модуля \textbf{requests}, которая последовательно считывает из командной строки адрес 
сервера, порт сервера, два целых числа \textbf{a} и \textbf{b}. Каждое значение вводится с новой строки. \\
После чего осуществляется GET-запрос на введенный адрес и порт с передачей параметров \textbf{a} и \textbf{b}.\\
То есть запрос вида: сервер:порт?a=значение_a&b=значение_b \\
Сервер возвращает ответ в формате \textbf{json}, в котором есть список с результатом целочисленного деления и остатком от целочисленного деления чисел \textbf{a} и 
\textbf{b} по ключу \textbf{result}, а также проверочная строка по ключу \textbf{check}. \\
Пример ответа для первого теста: \\
\{ \\
    'check': 'b0f66adc83641586656866813fd9dd0b8ebb63796075661ba45d1aa8089e1d44', \\ 
    'result': [8, 6] \\
\}

Программа должна вывести на первой строке значение списка \textbf{result}, отсортированные по возрастанию и разделенные 
пробелом. На второй строке необходимо вывести проверочную строку.

\InputFile
Строка - адрес сервера. \\
Целое число - порт сервера. \\
Целое число - число \textbf{a}. \\
Целое число - число \textbf{b}.
\OutputFile
Целые числа через пробел, отсортированные по возрастанию, из списка по ключу \textbf{result} в ответе сервера.
Проверочная строка из ответа сервера.
\Examples
\begin{example}
\exmp{http://127.0.0.1
7777
4
2}{0 2
b0f66adc83641586656866813fd9dd0b8ebb63796075661ba45d1aa8089e1d44}
\exmp{http://127.0.0.1
7777
-7777
888}{-9 215
077bdb2c3714897d4153c8b48245afaa5b8ccb64bf4c5c316638818fa3a22a09}
\end{example}\end{problem}
